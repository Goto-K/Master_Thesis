% !TEX root = ../MasterThesis_goto_v1.tex

\chapter{まとめと今後の展望} \label{chap:Conclusion}

本研究では, 深層学習を用いた崩壊点検出アルゴリズムの開発と実装を行った。
ジェットの再構成は, 加速器実験における新物理の手掛かりを探索する上で非常に重要な役割を果たしている。
ILCにおいては, ボトム・フレーバーとチャーム・フレーバーのジェットの高性能な分離が期待され, 現在はLCFIPlusのジェット再構成アルゴリズムが使用されている。
事象再構成アルゴリズムにはパターン認識技術が使用される場合も多く, 深層学習のような機械学習技術の発展にる物理解析の精度向上が期待されている。
しかし, 崩壊点検出アルゴリズムについては深層学習を用いたアルゴリズムの研究は殆ど行われておらず, 本研究がその先駆けとなっている。

深層学習技術として一般的なフィードフォワードニューラルネットワークに加え, リカレントニューラルネットワークの一種であるLSTMを独自に拡張したネットワークやAttentionなど最新のテクニックを活用した。
崩壊点検出アルゴリズムの実現に当たって二種類のネットワーク「1. 飛跡対についてのネットワーク」, 「2. 任意の数の飛跡についてのネットワーク」をTensorflow/Kerasを用いて実装した。
ネットワークのための訓練データとして$\rm e^+e^- \to b \bar{b}/c \bar{c}$事象のMCシミュレーションデータを使用し, 正解ラベルの作成にはMC情報やLCFIPlusのフィッターによる計算値を用いた。

これらのネットワークを組み合わせたアルゴリズムを作成し, 性能の評価を飛跡段階での効率を用いて行った。
性能の可否についてはLCFIPlusと比較する事によって評価し, まず崩壊点検出アルゴリズム単体での比較を行った後, ジェット再構成の最後段であるフレーバータギングでの比較を行った。
フレーバータギングの比較のため, 本ネットワーク・アルゴリズムのLCFIPlus, Marlinへの実装を行った。
本アルゴリズムはLCFIPlusのアルゴリズムと置換可能な状態であり, iLCSoft内での深層学習の活用について示すことができた。

崩壊点検出アルゴリズム単体での比較において, 本研究のアルゴリズムはsecondary vertexの再構成効率においてLCFIPlusのアルゴリズムを超えることができた (表\ref{PerformanceofAllEvents})。
一方で, 誤識別であるPrimary, Othersの値が悪化してしまっており単純な性能比較では性能の優劣を決めることが出来なかった。
フレーバータギングの比較において, 