% !TEX root = ../MasterThesis_goto_v1.tex

\begin{center}
\thispagestyle{empty}
{\Large 深層学習を用いたILC崩壊点検出アルゴリズムの開発}\\
九州大学大学院 理学府 物理学専攻 \\ 粒子物理学分野 素粒子実験研究室 \\
後藤\ 輝一 \\[1ex] 指導教員\ 末原\ 大幹,\ 川越\ 清以\\   \\
{\huge 概要}\\
\end{center}

本研究では人工知能技術の一つである深層学習を用いて, 国際リニアコライダー (ILC) の為の崩壊点検出アルゴリズムの開発を行なった。
崩壊点検出アルゴリズムはジェット再構成の最初段であり, 荷電粒子の飛跡から重いフレーバーのクォークの崩壊点を探索するアルゴリズムである。
ILCのジェット再構成は崩壊点検出アルゴリズムの後, ジェットクラスタリング, フレーバータギングと続く。
現在, ILCではジェット再構成にLCFIPlusというソフトウェアが使用されている。
LCFIPlusはiLCSoftというソフトウェアエコシステムの一つである。
LCFIPlusではフレーバータギングにのみ, 機械学習技術の一つであるBoosted Decision Trees (BDTs) が使用されており, その他のアルゴリズムは人の手によって定められた閾値によるカットベースな手法が使用されている。

近年, 粒子物理学の分野では物理解析やシミュレーションに深層学習を使用し, 精度や性能を改善する取り組みが行われている。
深層学習とは教師あり学習の一つであり, 分類や回帰問題を解く機械学習技術の一つである。
事象再構成についても, この深層学習を用いた性能の改善が期待されており, 本研究はその様な取り組みの一つである。
深層学習は取り扱うデータや問題の性質によって様々な発展的手法が提案されており, 本研究では特に言語や音声といった系列データを取り扱う回帰型ニューラルネットワーク (リカレントニューラルネットワーク) とAttention機構を使用した。
リカレントニューラルネットワークでは, 既存のネットワーク構造をそのまま使うのではなく独自のネッ トワーク構造を構築した。
深層学習の訓練データとしてILCの検出器コンセプトの一つであるInternational Large Detector (ILD) の検出器フルシミュレーションデータを使用したが, 本研究のアルゴリズムの基本的な発想は検出器の違いによらず使用できる。

本論文では, 深層学習を用いた崩壊点検出アルゴリズムの開発とアルゴリズムのLCFIPlusへの導入をまとめた。
LCFIPlus内の機能はMarlinというアプリケーションフレームワークのプロセッサーとして運用されている。
ここでは本アルゴリズムをLCFIPlusフレームワーク内に実装し, LCFIPlusのアルゴリズムと置換可能にした。
また, 本アルゴリズムについての性能はLCFIPlusのアルゴリズムと比較する事によって評価した。
評価指標として飛跡段階での崩壊点の再構成効率を用いた。
LCFIPlusの崩壊点検出アルゴリズムと比較して誤認識率の高い結果となったが, 一方で再構成効率の良い結果を得ることができた。