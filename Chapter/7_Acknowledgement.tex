% !TEX root = ../MasterThesis_goto_v1.tex

\clearpage

\chapter*{謝辞} \label{sec:Acknowledgement}

本研究は過去に例のない研究テーマであり, 様々な方のご協力, ご助言の元, 遂行することが出来ました。
平時とは異なる環境の中, 修士論文の執筆に至ることが出来たのは皆様のお力添えによるものであると考えております。

特に指導教員である末原大幹助教には, 研究内容から研究生活に至るまで様々な面でサポートして頂き, 海外への出張や国際会議での発表など得難い経験をさせて頂きました。私の興味や希望に合ったテーマを選んで頂き, 非常に楽しく研究に取り組む事ができました。
末原助教のお力添え無くしては, 本研究を成し遂げることは出来なかったと痛感しております。誠にありがとうございます。
川越清以教授におかれましては, お忙しい中, 学会の発表練習や進捗報告, ILCに関するゼミなどで多くのご助言を頂きました。更に, 前例のない挑戦的な研究テーマに挑ませて頂き, 誠にありがとうございます。
吉岡瑞樹准教授には, 講義のTA業務やILC関連の会議の運営などでお世話になりました。
東城順次准教授には, 論文紹介などでご見識を賜りました。ありがとうございます。
織田勧助教におかれましては, 同じく論文紹介において私の興味分野に関心を持って頂きました事, 心より感謝申し上げます。
音野瑛俊助教, 山中隆志助教には研究室運営など様々な面でお世話になりました。
小林大特任助教には, 研究内容から日常的な生活まで様々なお話を聞いて頂きました。また, 比較的世代の近い研究者としての姿から研究に取り組む姿勢などを学ばせて頂きました。
重松さおり氏には, 例年とは異なる状況で研究を行うにあたって必要な物品や環境などを迅速に手配して頂き, お陰様で研究に継続して取り組む事ができました。

倉田正和研究員 (現東京大学研究員) には, 短い期間ではありましたが, 同じ研究内容に取り組む研究者としての貴重なご意見を頂きました。
東京大学の加藤悠氏には, ILCグループに配属して頂いた直後からお世話になり, 研究に必要な技術や常識を一から教えて頂きました。

ILCグループの先輩である出口遊斗氏, 上杉悠人氏には, 研究のみならず, 就職活動や出張の手続きなど様々な面で相談に乗って頂きました。
同じくILCグループの後輩である久原真美さんには, Monthly Meetingなどの進捗報告で深層学習という馴染みの無い分野の発表をさせてしまい, ご面倒をお掛けしました。

研究室の先輩である中居勇樹氏には, 学部生の頃の実験を見て頂き, 常に適切なご助言を頂きました。
同じく研究室の先輩である高田秀佐氏, 古賀淳氏, 山口尚輝氏, 宮崎祐太氏, 野口恭平氏, 竹内佑甫氏には日々の四方山話から研究の問題まで色々なお話を聞いて頂きました。
特に, 古賀氏と宮崎氏には本論文を執筆するにあたって様々なご意見を頂き, 参考にさせて頂きました。

同期である来見田将大氏は, 出不精な私を色々な場所に連れて行って下さいました。機会があれば, もう一度釣りに行き, 今度はちゃんとした釣果を得たいと思っています。
荘司大志氏には, 同じ加速器実験をしている同期として様々な事を相談させて頂きました。また, 食事に対するセンスなど見習うところが多く, 勉強させて頂きました。
松本岳氏には, 学部生の実験の頃より, ハードウェアに関する技術力をとても頼りにさせて頂きました。
矢野浩大氏には, 日頃より私の毒にも薬にもならない話を延々と聞いて頂き色々とご迷惑をお掛けした様に思います。学会発表でもwi-fiを貸して頂き助けられました。

研究室の後輩である姚君, 松崎君, 嶋津君, 岩下君, 川上君, 津村君, 樋口君, 松原君の研究に取り組む姿は, 自身の研究の刺激となりました。

最後となりますが, ここまで私を応援し, 育てて下さった両親に心より感謝の意を示し, 本論文の結びとさせて頂きます。






