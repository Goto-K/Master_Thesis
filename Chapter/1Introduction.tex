% !TEX root = ../MasterThesis_goto_v1.tex

%%%%%%%%%%%%%%%%%%%%%%%%%%%%%%%%%%%%%%%%%%%%%%%%%%%%%%%%%%%%%%%%%%%%%%%%%%%%%%%%%%%%%%%%%%%%%%%%%%%%%
\chapter{序論} \label{chap:Introduction}

本章では、まず\ref{Intro:StandardModel}節で素粒子の振る舞いを記述する理論である、標準模型 (Standard Model, SM) について紹介する。

次に\ref{Intro:InternationalLinearColliderProject}節にて、この標準模型や標準模型を超える物理 (Physics beyond the Standard Model, BSM) を探索するための国際リニアコライダー (International Linear Collider, ILC) 計画についての説明を行う。
そのようなILCが目指す物理について\ref{Intro:PhysicsofILC}節で述べる。
また、ILCで使用される予定の検出器やソフトウェアに関する説明を\ref{Intro:InternationalLargeDetector}節と\ref{Intro:SoftwareandEventReconstructionofILC}節で行う。

更に本研究の目的について\ref{Intro:Purpose}節で、本論文の流れについて\ref{Intro:Flow}節でそれぞれ述べ本論文の序論とする。

本章の作成にあたり、参考文献\cite{GlobalProject, InterimDesignReport}を使用した。


%%%%%%%%%%%%%%%%%%%%%%%%%%%%%%%%%%%%%%%%%%%%%%%%%%%%%%%%%%%%%%%%%%%%%%%%%%%%%%%%%%%%%%%%%%%%%%%%%%%%%
\section{標準模型} \label{Intro:StandardModel}

宇宙の誕生や、生物の発生と同様に、物質の起源は人類の根元的な問いの一つである。
そのような物質を構成する最小の粒子のことを素粒子といい、その素粒子の振る舞いを記述する理論の一つが標準模型である。
標準模型によると、素粒子はスピン半整数のフェルミ粒子とスピン整数のボース粒子に分類される。

標準模型ではフェルミ粒子は全てスピン$1/2$の粒子で構成され、陽子や中性子などのハドロンを構成するクォークと電子やニュートリノなどのレプトンに分けられる。
更に、クォークは電荷が$+2/3$のアップクォーク系列と$-1/3$のダウンクォーク系列に、レプトンは電荷が$-1$の荷電レプトンと中性電荷の中性レプトン (ニュートリノ) に分けられる。
また、それぞれ世代と呼ばれるものを構成し現在は3つの世代が確認されている。
クォークの場合はアップクォーク$\textit u$、チャームクォーク$\textit c$、トップクォーク$\textit t$、ダウンクォーク$\textit d$、ストレンジクォーク$\textit s$、ボトムクォーク$\textit b$が存在する。
レプトンの場合は荷電レプトンとして、電子$\textit e^-$、ミュー粒子$\it \mu^-$、タウ粒子$\it \tau^-$、中性レプトンとして、電子ニュートリノ$\it \nu_e$、ミューニュートリノ$\it \nu_{\mu}$、タウニュートリノ$\it \nu_{\tau}$が存在している。
これらの系列や世代間の違いをフレーバーと呼んでいる。

ボース粒子は基本的な四つの相互作用である、強い相互作用、弱い相互作用、電磁相互作用、重力相互作用の内、重力相互作用を除いた三つの相互作用をそれぞれ媒介するスピン$1$のゲージ粒子と、対称性を破り素粒子に質量を与えるスピン$0$、中性電荷のヒッグス粒子$\rm H$で構成される。
電磁相互作用を媒介する粒子として、中性電荷の光子$\rm \gamma$、強い相互作用を媒介する粒子として、中性電荷のグルーオン$\rm g$、弱い相互作用を媒介する粒子として、電荷$\pm 1$のWボソン$\rm W^{\pm}$、中性電荷のZボソン$\rm Z$が存在する。

更にフェルミ粒子には、質量やスピンが等しく電荷の正負のみが反転した反粒子が存在する。
これら反粒子と通常の粒子が衝突すると対消滅が起こり、その質量が全てエネルギーへと変換される。
一方、ある粒子の質量の二倍以上のエネルギーを生じさせた場合は対生成が起こり、その粒子と反粒子の対が生成されることがある。

以上の素粒子を標準模型の素粒子といい、一般に図\ref{1SMParticle}のように纏められている。

\begin{figure}[htbp]
 \centering
 \includegraphics[width=0.9\textwidth]{Figure/1Introduction/1SMParticle.png}
 \caption{標準模型の素粒子}
 \label{1SMParticle}
\end{figure}

標準模型は様々な実験で非常によく確かめられているが、ダークマターをはじめとする幾つかの物理現象を説明できていない。
これらの標準模型で説明できない物理現象を標準模型を超える物理 (BSM) といい、現在は様々な実験施設でBSMの探索が行われている。
次節のILC計画はそのような試みの一つである。


%%%%%%%%%%%%%%%%%%%%%%%%%%%%%%%%%%%%%%%%%%%%%%%%%%%%%%%%%%%%%%%%%%%%%%%%%%%%%%%%%%%%%%%%%%%%%%%%%%%%%
\section{ILC計画} \label{Intro:InternationalLinearColliderProject}

ILC計画とは、日本の東北にある北上山地に図\ref{2InternationalLinearCollider}のような全長$20.5\ \mathrm{km}$の国際リニアコライダー (ILC) を建設する計画である。
ILC計画は国際共同研究であり、2013年に出版されたThe Technical Design Report (TDR\cite{ILCTDRVES}) には2400人の研究者、48ヶ国、392の大学と研究機関のグループが著名している。
また、技術開発はリニアコライダーコラボレーション (The Linear Collider Collaboration, LCC) によって推進され、LCCの活動は国際将来加速器委員会 (The International Committee for Future Accelerator, ICFA) の下、リニアコライダー国際推進委員会 (Linear Collider Board, LCB) によって監督されている。
2021年現在、ILC計画は準備段階へ向けて計画が進められており、ILC準備研究所 (ILC Pre-Lab) の為の準備としてICFAはILCの国際推進チーム (International Development Team, IDT) の設立を承認した。
今後はLCCやLCBに代わり、このILC国際推進チームが2020年8月よりILC計画の推進を行なっている。
ILC計画の今後の大まかな流れを図\ref{3ILCProject}に示す。

\begin{figure}[htbp]
 \centering
  \includegraphics[trim = 0 100 0 100, width=0.9\textwidth, clip]{Figure/1Introduction/2InternationalLinearCollider.jpg}
  \caption[ILCの全体像]
                 {ILCの全体像\cite{ILCPHOTO}}
  \label{2InternationalLinearCollider}
\end{figure}

\begin{figure}[htbp]
 \centering
 \includegraphics[width=0.85\textwidth]{Figure/1Introduction/3ILCProject.png}
 \caption[ILC計画の今後]
                {ILC計画の今後\cite{RecommendationsonILCProjectImplementation}}
 \label{3ILCProject}
\end{figure}


%%%%%%%%%%%%%%%%%%%%%%%%%%%%%%%%%%%%%%%%%%%%%%%%%%%%%%%%%%%%%%%%%%%%%%%%%%%%%%%%%%%%%%%%%%%%%%%%%%%%%
\section{ILCの物理} \label{Intro:PhysicsofILC}

ヒッグス粒子が2012年に欧州原子核研究機構 (CERN) の大型ハドロン衝突型加速器 (Large Hadron Collider, LHC) で発見されて以降、ヒッグス粒子の性質について、より詳細な調査が行われている。
ヒッグス機構は標準模型の中で電弱相互作用の対称性を破り、素粒子に質量を与える役割を担っており、更に質量に比例した結合定数を持つという特徴を持っている。
このような性質からヒッグス粒子の振る舞いは標準模型によって詳細に決定される。
また、BSMによって標準模型の素粒子の振る舞いに変化が生じた場合、ヒッグス粒子はその影響を受けると予想されている。
特にヒッグス粒子の結合定数は、そのようなBSMの模型やパラメータによって異なった変化する。

ILCは、このヒッグス粒子の性質を詳細に調べる為のヒッグスファクトリーとしての役割を期待されている。
LHCが陽子-陽子を衝突させる加速器であるのに対し、ILCは電子-陽電子を衝突させる加速器である。
これは粒子と反粒子の関係となっており、ILCはLHCと比較して目的とする事象に対しエネルギーをより効率的に使うことができる。
更に電子-陽電子衝突の場合は相互作用の関係から、背景事象が少ないという特徴を持っている。

ILCは電子・陽電子衝突によって、Z粒子とヒッグス粒子を生成する$\rm e^+e^- \to Zh$事象の反応断面積が最大となる重心系エネルギー$\sqrt{s}=250\ \mathrm{GeV}$での運転開始 (ILC250) を予定している(図\ref{4eetoZH})。
また、ILCには様々な物理目標を達成する為に多数のアップグレードオプションが存在し、重心系エネルギーについては主線形加速器を延長、加速勾配を上昇することで$1\ \mathrm{TeV}$以上への拡張が可能である。

\begin{figure}[htbp]
 \centering
 \includegraphics[trim= 0 50 0 50, width=0.9\textwidth, clip]{Figure/1Introduction/4eetoZH.png}
 \caption[重心系エネルギーとヒッグス事象生成断面積の関係]{重心系エネルギーとヒッグス事象生成断面積の関係\cite{ILCTDRVP}。ヒッグス粒子の質量を$125\ \mathrm{GeV}$とした時の$\rm e^+e^- \to Zh$事象の生成断面積を赤線でプロットしている。また、WW fusion $\rm e^+e^- \to \nu \bar{\nu} h$事象、ZZ fusion $\rm e^+e^- \to e^+ e^- h$事象をそれぞれ青線、緑線で表現している。}
 % The International Linear Collider Technical Design Report - Volume 2: Physics Fig 2.7
 \label{4eetoZH}
\end{figure}

$\rm e^+e^- \to Zh$事象はヒッグス粒子の反跳粒子である$\rm Z$粒子を再構成することによって、ヒッグス粒子の崩壊モードに寄らずヒッグス粒子の四元運動量を測定できるという点で非常に重要である。
また、背景事象である$\rm e^+e^- \to Z\gamma$や$\rm e^+e^- \to ZZ$に関してもよく理解されており、電弱相互作用の計算によって不定性を$0.1\ \%$程度に抑えることができる\cite{GlobalProject}。
この反跳粒子を使用した解析では$\rm e^+e^- \to Zh$事象の全断面積を得ることができ、絶対正規化されたヒッグス粒子の結合定数やヒッグス粒子の暗黒物質への崩壊についての測定が可能となる。

$\rm e^+e^- \to Zh$事象での終状態の$\rm Z$粒子は、レプトン対 ($\rm Z \to l^+l^-$) またはクォーク対 ($\rm Z \to q\bar{q}$) に崩壊する。
レプトン対にはおよそ$30\%$程度 (荷電レプトンへ$10\%$・ニュートリノへ$20\%$) の割合で崩壊し、クォーク対には残りの$70\%$程度の割合で崩壊する為、統計量を大きくするという点でクォーク対をより精度よく識別することは非常に重要である。
これらのクォーク対はエネルギー効率のために、それぞれ真空中でクォークの粒子反粒子対を生成・結合しハドロンとなる。
この過程で生成されたクォークも同様にハドロンを形成するため、初めのクォーク対のそれぞれの進行方向には多数のハドロン粒子が生成されることとなる。
これをジェットといい、ヒッグス粒子に関してもクォーク対への崩壊が考えられ、それらのクォーク対は同様にジェットを形成する。
特に$\rm H \to b \bar{b},\ c \bar{c},\ g \bar{g}$の識別ではジェットの元となるクォークのフレーバーを高精度に同定する必要があり、この様なジェットの再構成については、\ref{Intro:SoftERILC:JetReconstruction}項にて説明する。


%%%%%%%%%%%%%%%%%%%%%%%%%%%%%%%%%%%%%%%%%%%%%%%%%%%%%%%%%%%%%%%%%%%%%%%%%%%%%%%%%%%%%%%%%%%%%%%%%%%%%
\section{ILCの検出器 -International Large Detector (ILD)-} \label{Intro:InternationalLargeDetector}

ILCではSilicon Detector (SiD) とInternational Large Detector (ILD) 二つの検出器コンセプトが検討されている。
本研究はILDの検出器シミュレーションデータを用いている為、本節ではILDについて簡単な解説を行う。
ただし、本研究の基本的な発想はそのような検出器の違いに寄らず使用することができる。

ILDはヒッグス粒子や電弱相互作用の物理からの要求値を満たすように設計され、また後述するParticle Flow (\ref{Intro:SoftERILC:ParticleReconstruction}項) によって最適化されている。
またILDは様々なサブディテクターによって構成され、ビームの衝突点 (図\ref{5-1InternationalLargeDetector}の右下) を包む様に内側から順に、Vertex Detector (VTX)、Silicon Internal Tracker (SIT)、Time Projection Chamber (TPC)、Electromagnetic Calorimeter (ECAL)、Hadron Calorimeter (HCAL)、Iron Yoke (Muon) が並んでいる。
また、HCALとIron Yokeの間にはSolenoid Coilがあり、$3.5 \mathrm{T}$の磁場をかけている。
ILDでは、VTXやSIT、TPCを用いて荷電粒子の飛跡を測定し、ECALによって電子や光子、HCALによってハドロン粒子のエネルギーを測定する。

衝突点の前方方向には、Forward Tracking Detector (FTD)、Luminosity Calorimeter (LumiCAL)、LHCAL、Beam Calorimeter (BeamCAL) が並んでいる。
それぞれのサブディテクターの詳細については表\ref{ILDSubdetectorParametersBarrel}と表\ref{ILDSubdetectorParametersEndCap}にまとめる。

\begin{figure}[htbp]
 \centering
 \begin{minipage}{1.0\textwidth}
  \centering
   \begin{minipage}{0.48\textwidth}
    \centering
    \includegraphics[width=1.0\textwidth, clip]{Figure/1Introduction/5-1InternationalLargeDetector.jpg}
    \subcaption{ILDの全体像\cite{ILCPHOTO}}
   \end{minipage}
   \begin{minipage}{0.48\textwidth}
   \centering
    \includegraphics[width=1.0\textwidth, clip]{Figure/1Introduction/5-2InternationalLargeDetector.png}
    % INTERIM DESIGN REPORT Figure 5.1
    \subcaption{ILDの縦断面\cite{InterimDesignReport}}
   \end{minipage}
  \end{minipage}  
  \caption{International Large Detector (ILD) }
  \label{5-1InternationalLargeDetector}
\end{figure}

\begin{table}[htbp]
 \centering
  \small
   \scalebox{0.9}{
  \begin{tabular}{l c c c l} \hline
     & $r_{in} \mathrm{[mm]}$ & $r_{out} \mathrm{[mm]}$ & $z_{max} \mathrm{[mm]}$ & 要素技術\\ \hline \hline
    VTX & 16 & 60 & 125 & シリコンピクセルセンサー\\
    SIT & 153 & 303 & 644 & シリコンピクセルセンサー\\
    TPC & 329 & 1770 & 2350 & マイクロパターンガス検出器\\
    SET & 1773 & 1776 & 2300 & シリコンストリップセンサー\\ \hline
    ECAL & 1805 & 2028 & 2350 & 吸収層 : タングステン\\
    &&&& センサー : シリコン/シンチレーター\\
    HCAL & 2058 & 3345 & 2350 & 吸収層 : スチール\\
    &&&& センサー : シンチレーター/RPCガス\\ \hline
    Coil & 3425 & 4175 & 3872\\
    Muon & 4450 & 7755 & 4047 & センサー : シンチレーター\\ \hline
  \end{tabular}
  }
  \caption[ILDサブディテクターの詳細なパラメータ (バレル)]{ILDサブディテクターの詳細なパラメータ (バレル) \cite{InterimDesignReport}}
  \label{ILDSubdetectorParametersBarrel}
\end{table}

\begin{table}[htbp]
 \centering
 \small
 \scalebox{0.9}{
  \begin{tabular}{l c c c c l} \hline
     & $z_{min} \mathrm{[mm]}$ & $z_{max} \mathrm{[mm]}$ & $r_{in} \mathrm{[mm]}$ & $r_{out} \mathrm{[mm]}$ & 要素技術\\ \hline \hline
    FTD & 220 & 371 & & 153 & シリコンピクセルセンサー\\
            & 645 & 2212 & & 300 & シリコンストリップセンサー\\ \hline
    ECAL & 2411 & 2635 & 250 & 2096 & 吸収層 : タングステン\\
    &&&&& センサー : シリコン/シンチレーター\\
    HCAL & 2650 & 3937 & 350 & 3226 & 吸収層 : スチール\\
    &&&&& センサー : シンチレーター/RPCガス\\
    Muon & 4072 & 6712 & 350 & 7716 & センサー : シンチレーター\\ \hline
    BeamCAL & 3115 & 3315 & 18 & 140 & 吸収層 : タングステン\\
    &&&&& GaAs読み出し \\
    LumiCAL & 2412 & 2541 & 84 & 194 & 吸収層 : タングステン\\
    &&&&& センサー : シリコン\\
    LHCAL & 2680 & 3160 & 130 & 315 &吸収層 : タングステン\\ \hline
  \end{tabular}
  }
  \caption[ILDサブディテクターの詳細なパラメータ (エンドキャップ)]{ILDサブディテクターの詳細なパラメータ (エンドキャップ) \cite{InterimDesignReport}}
  \label{ILDSubdetectorParametersEndCap}
\end{table}


%%%%%%%%%%%%%%%%%%%%%%%%%%%%%%%%%%%%%%%%%%%%%%%%%%%%%%%%%%%%%%%%%%%%%%%%%%%%%%%%%%%%%%%%%%%%%%%%%%%%%
\section{ILCのソフトウェアと事象再構成} \label{Intro:SoftwareandEventReconstructionofILC}

ここではILCで使用されるソフトウェアと事象再構成について述べる。
事象再構成とは、加速器実験によって得られるデータから飛跡やジェットなどの物理情報を再構成するアルゴリズムである。
そのような再構成は電子-陽電子の衝突毎に行われ、この衝突一回分の事を事象という。

ILCのソフトウェアはiLCSoft\cite{iLCSoft}と呼ばれるソフトウェアエコシステムにまとめられている。

ILCにおける事象再構成は、トラッキングやParticle Flowといった\ref{Intro:SoftERILC:ParticleReconstruction}. 粒子の再構成と、更にそれらによって再構成された粒子を使った\ref{Intro:SoftERILC:JetReconstruction}. ジェットの再構成に区分できる。
ILCではジェットの再構成は崩壊点検出、ジェットクラスタリング、フレーバータギングという行程で行われる。
これらジェットの再構成はiLCSoft内のLCFIPlus\cite{LCFIPlus}が使用されている。


%%%%%%%%%%%%%%%%%%%%%%%%%%%%%%%%%%%%%%%%%%%%%%%%%%%%%%%%%%%%%%%%%%%%%%%%
\subsection{ソフトウェア} \label{Intro:SoftERILC:Software}

ILCは実際の実験データを取得できないため、本研究で使用するデータは全てシミュレーションデータである。
シミュレーションデータは標準模型とBSMを用いて、モンテカルロ (Monte Carlo, MC) 法によって生成されている。
それらのシミュレーションデータはLCIOと呼ばれる階層型のEvent Data Model (EDM) によって管理されている。
LCIOでは、MC情報から事象の生データ、デジタル化、解析や後述する再構成に至るまでが紐づけられており、階層的に取り扱うことができる。

ILCのソフトウェアは二つの検出器コンセプト (ILD, SiD) で共通しており、前述したように現在はiLCSoftというソフトウェアエコシステムによって統括されている。
提供されているAPIの言語はC++・java・Fortranである。

それらソフトウェアモジュールはMarlin\cite{Marlinpaper}というC++アプリケーションフレームワークによって運用されており、プロセッサーと呼ばれるモジュールを作成・組み込むことにより、様々な再構成・解析アルゴリズムを簡単に別のモジュールへ置き換えることができる。
また、データの入出力はLCIO、ROOTフォーマットによって行われる。


%%%%%%%%%%%%%%%%%%%%%%%%%%%%%%%%%%%%%%%%%%%%%%%%%%%%%%%%%%%%%%%%%%%%%%%%
\subsection{飛跡の再構成} \label{Intro:SoftERILC:ParticleReconstruction}

飛跡の再構成は、シミュレーションされた検出器のデータから、粒子 (飛跡) を再構成するトラッキングと、そのようにして得られた個々のトラッキング検出器 (VTX, SIT, TPCなど) の飛跡や粒子についての情報を繋ぎ合わせ、より高精度の粒子情報を提供するParticle Flowという手順によって行われる。

トラッキングではKalman-Filterが使用され、まず荷電粒子の軌跡をパターン認識を用いて再構成し、次にそれらの軌跡について運動学的な物理量をフィッティングによって抽出している。
ILDにおいて異なるサブディテクターのトラッキングは異なるアルゴリズムが使用されている。

トラッキングによって運動学的な物理量を得られるのは荷電粒子のみである。
中性粒子はVXDやTPCに飛跡を残さない為、他のサブディテクターを用いた再構成が必要である。
このように粒子の性質によって、再構成を行うべき最適なサブディテクターは異なる。
ILCにおける粒子種毎の最適な測定手法や再構成手法の選択は、Particle Flowによって提供されている。
Particle Flowでは荷電粒子をトラッキング検出器によって測定し、光子や中性ハドロンの再構成をそれぞれECALやHCALで行う。
これらの再構成について、iLCSoftではPandoraPFAと呼ばれるアルゴリズムが使用されている。
このアルゴリズムでは、まずカロリメータのヒットをクラスター化し、それらクラスターとトラッキング情報を関連づけ粒子識別を行っている。

以上が飛跡の再構成である。
飛跡の再構成では、検出器で得られた情報から粒子を再構成するまでを行なっている。
実際には。注目すべき物理事象は前述したジェットのような特徴的なシグネチャを残す為、次項のジェットの再構成による更なる事象再構成が必要である。


%%%%%%%%%%%%%%%%%%%%%%%%%%%%%%%%%%%%%%%%%%%%%%%%%%%%%%%%%%%%%%%%%%%%%%%%
\subsection{ジェットの再構成} \label{Intro:SoftERILC:JetReconstruction}

事象中に生じたクォークは\ref{Intro:PhysicsofILC}節で述べたようにジェットを形成する。
ジェットには多数の粒子 (飛跡) が含まれ、それら多数の粒子の親となる粒子が崩壊した点を崩壊点 (Vertex) という。
特に、ハドロンのような準安定な親粒子の崩壊点の事をsecondary vertexといい、電子-陽電子の衝突点をprimary vertexという (図\ref{6ReconstructedVertex})。
ジェットの再構成では、まずこの崩壊点を崩壊点検出 (Vertex Finder) を用いて探索し、そのようにして得られた崩壊点を用いてジェット中の粒子を分離するジェットクラスタリング (Jet Clustering) が行われ、そのジェット中の飛跡から崩壊した準安定な親粒子のクォーク・フレーバーを識別するフレーバータギング (Flavor Tagging) が行われる。

\begin{figure}[htbp]
 \centering
 \includegraphics[trim = 0 100 0 50, width=1.0\textwidth, clip]{Figure/1Introduction/6ReconstructedVertex.png}
 \caption[primary vertexとsecondary vertexの図示]{primary vertexとsecondary vertexの図示。左右から電子・陽電子ビームが入射され、図中央で衝突したと仮定している。灰色の破線は準安定なハドロンを表現しており、赤線の飛跡と共にIP (primary vertex) から生じている。ハドロンは更に図右上と左下で崩壊し、secondary vertexを残す。青線はこのsecondary vertex由来の飛跡である。}
 \label{6ReconstructedVertex}
\end{figure}

崩壊点検出では、二本以上の飛跡について交点を求めるフィッティングを用いている。
この時、フィッティングの健全性は$\chi^2$値によって把握している。

まずprimary vertexの再構成をTear-Down法によって行う。
具体的には予想されるビームスポットと事象中の全飛跡についてフィッティングを行い、$\chi^2$値が一定以下になるまで$\chi^2$値への寄与が大きい飛跡を一本ずつ取り除くことによって、残った飛跡をprimary vertex由来であると判定している。
次にsecondary vertexの再構成をBuild-Up法によって行う。
ここでは、primary vertexに含まれていない飛跡について、二本の飛跡 (飛跡対) の全ての組み合わせを作りフィッティングを行う。
そのようにして得られた$\chi^2$値と運動量の方向、不変質量などをカットベースに判定し、secondary vertexについての飛跡対を選別する。
更に、この飛跡対について飛跡を加えていくことでsecondary vertexを再構成している。


崩壊点検出で得られた崩壊点を用いて、ジェットクラスタリングが行われる。
ジェットクラスタリングではDurhamアルゴリズム\cite{Durhampaper}を使用し、事象中の中性粒子を含めたより多くの情報を用いてジェット中の粒子をクラスター化している。

また、そのようにクラスター化されたジェット中の粒子についてフレーバータギングが行われる。
フレーバータギングではBoosted Decision Trees (BDTs) を用いて親粒子のフレーバーを識別している。

以上がジェットの再構成である。
様々な物理解析において、ジェットの個数やそのフレーバーの識別は信号事象と背景事象の区別や物理解析などに使用されている。
したがって、ジェットの再構成の性能向上はあらゆる物理解析の性能向上と直結していると言える。


%%%%%%%%%%%%%%%%%%%%%%%%%%%%%%%%%%%%%%%%%%%%%%%%%%%%%%%%%%%%%%%%%%%%%%%%%%%%%%%%%%%%%%%%%%%%%%%%%%%%%
\section{本研究の目的} \label{Intro:Purpose}

本研究の目的は、深層学習を使用して\ref{Intro:SoftERILC:JetReconstruction}項で紹介した崩壊点検出を開発・改善することである。
ILCでは現在LCFIPlus内の崩壊点検出が使用されているが、primary vertexやsecondary vertexの選別に人が定めた閾値が多く含まれており、カットベースな評価が行われている。
このような人が定めた閾値は最適ではなく、何らかの情報を欠損してしまっている可能性がある。
本研究では深層学習を用いたパターン認識の技術から、新しい崩壊点検出アルゴリズムを提案しより柔軟な識別を行うことを目標とする。

また、この研究は深層学習を用いて事象再構成を改善していくプロジェクトの一つである。
最終的には概念図\ref{7JetReconstructionwithDeepLearning}に示すように、全ての事象再構成アルゴリズムを深層学習や機械学習に置き換えることを目指している。
これまでILCの事象再構成ではあまり深層学習は使われておらず、特に崩壊点検出に関しては本研究が殆ど初めての試みである。

\begin{figure}[htbp]
 \centering
 \includegraphics[trim = 0 100 0 50, width=1.0\textwidth, clip]{Figure/1Introduction/7JetReconstructionwithDeepLearning.png}
 \caption[深層学習によるジェットの再構成]{深層学習によるジェットの再構成。図左は現行の手法、図右は目標としている深層学習を用いた再構成手法の概念図である。現行の手法では多くの場合、数値計算やそこから得られる変数に閾値を課してカットベースな評価が行われている。現在は深層学習への置き換えの第一段階として、それぞれの役割に特化したサブモデルと入れ替えを行い性能の改善を図っている。}
 \label{7JetReconstructionwithDeepLearning}
\end{figure}

したがって、本研究のソフトウェア開発としての目的は、LCFIPlusへの導入や次世代のLCFIPlusへの深層学習実装における起点となる事である。
ここでは、深層学習の実装・構築からiLCSoftへの導入を行い、ILC研究における深層学習導入の先駆けとなることを目指す。


%%%%%%%%%%%%%%%%%%%%%%%%%%%%%%%%%%%%%%%%%%%%%%%%%%%%%%%%%%%%%%%%%%%%%%%%%%%%%%%%%%%%%%%%%%%%%%%%%%%%%
\section{本論文の流れ} \label{Intro:Flow}

本章と\ref{chap:DeepLearning}章は本論文の導入である。

\ref{chap:DeepLearning}章では本論文の核となる技術である深層学習について解説を行う。
ここでは、本研究を理解する為に必要な技術領域や背景理論について簡潔な導入を行い、\ref{chap:Networks}章以降では、この\ref{chap:DeepLearning}章の内容を前提とした議論を行う。
ただし深層学習に関しての種々のテクニックについては経験則によるものが多いため、本研究で使用したものについては\ref{chap:DeepLearning}章では説明せず、\ref{chap:Networks}章で述べる。
また、具体的な実装に関しても同様に\ref{chap:DeepLearning}章では記載せず、付録\ref{sec:Code}にまとめる事とする。\\

\ref{chap:Networks}章と\ref{chap:VertexFinderwithDL}章、\ref{chap:Comparison}章は本論文の本題である。

\ref{chap:Networks}章では本研究で使用するデータと作成した深層学習のネットワークについて、その構造の解説や評価を行う。
また、深層学習を使用した崩壊点検出についての発想や計算環境に関しても\ref{chap:Networks}章で述べる。

\ref{chap:VertexFinderwithDL}章では\ref{chap:Networks}章で作成したネットワークを用いた崩壊点検出について、アルゴリズムや各種最適化、簡単な評価を行う。

\ref{chap:Comparison}章ではその様にして得られた崩壊点検出と、LCFIPlusで使用されている現行の崩壊点検出との比較について述べる。\\

\ref{chap:Conclusion}章は本論文の結論である。
ここでは、本研究をまとめると共に今後の展望について述べる。

















