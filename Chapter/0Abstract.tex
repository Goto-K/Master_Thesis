% !TEX root = ../MasterThesis_goto_v1.tex

\begin{center}
\if 0
\thispagestyle{empty}
{\Large 修士論文テンプレート}\\
九州大学大学院 理学府 物理学専攻 \\ 粒子物理学分野 素粒子実験研究室 \\
素粒子\ 実験 \\[1ex] 指導教員\ 氏\ 名\\   \\
\fi
{\huge 概要}\\
\end{center}

素粒子はそれが従う統計によって二種類に分類され、フェルミ統計に従う粒子をフェルミ粒子、ボース統計に従う粒子をボース粒子と呼ぶ。現時点で存在が知られているフェルミ粒子はクォークとレプトンとに分類される。一方、現時点で知られているボース素粒子には、素粒子間の相互作用を伝達するゲージ粒子と、素粒子に質量を与えるヒッグス機構に関連して現れるヒッグス粒子とがある。ゲージ粒子のうち、重力を媒介するとされる重力子は未発見である。
素粒子の大きさは分かっておらず、大きさが無い(点粒子)とする理論と、非常に小さいがある大きさを持つとする理論がある。
標準模型(標準理論)では素粒子には大きさが無い(点粒子)ものとして扱っており、現時点では実験結果と矛盾が生じていない。ただし、点粒子は空間が最小単位の存在しない無限に分割可能な連続体であることを前提としているが、標準模型で扱うスケールより15桁以上小さいスケール(プランク長スケール)においては、空間が連続的であるか離散的であるかは判明していない。離散的である場合には点粒子として扱えない。
超弦理論においては全ての素粒子は有限の大きさを持つひもの振動状態であるとされる。
我々が普段目にする物質は(微小な、あるいは大きさが無い)素粒子からできているにも関わらず、有限の大きさを持っている。それは、複数の素粒子が運動する有限の領域が、ハドロンや原子などの大きさを持つ粒子を構成することによる。
素粒子のうちほとんどのものは、自然界に単独で安定的に存在しているわけではないので、宇宙線の観測や加速器による生成反応により発見・研究された。素粒子の様々な性質を実験で調べ、それを理論的に体系化していくこと、及び理論的に予言される素粒子を実験で探索していくことが、素粒子物理学の研究目的である。