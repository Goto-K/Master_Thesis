% !TEX root = ../MasterThesis_goto_v1.tex

%%%%%%%%%%%%%%%%%%%%%%%%%%%%%%%%%%%%%%%%%%%%%%%%%%%%%%%%%%%%%%%%%%%%%%%%%%%%%%%%%%%%%%%%%%%%%%%%%%%%%
\chapter{現行の手法との比較} \label{chap:Comparison}

本研究での崩壊点検出の性能の可否についての判断は現行の手法と比較することによって行う。
本章では、そのような比較について議論を行う。
まず\ref{Com:ComparisonwithVF}節では崩壊点検出単体での性能の比較をする。

崩壊点検出はジェットの再構成における第一段階である。
したがって、最終的な性能を知るためには本研究で作成した崩壊点検出を用いたフレーバータギングの性能を確認せねばならない。
このようなフレーバータギングの性能を確認する為、ネットワークのLCFIPlusへの導入を行った。
LCFIPlusは前述したようにiLCSoftの一部であり、C++で記述されている。
一方、深層学習のネットワークはPythonで記述しており、直接LCFIPlusへ導入することは不可能である。
その為、ネットワークの推論部分をC++へ移行し、Tensorflowなど種々のソフトウェアをcmake環境で動作するよう再構築を行った。
詳細な実装方法については私のGitHubにまとめている\cite{GitHubGotoKLCFIPlus}。
このようなC++へ移行された崩壊点検出の動作環境を表\ref{SoftwareEnvironments}に示す。

\begin{table}[htb]
 \centering
 \small
 \scalebox{0.8}{
  \begin{tabular}{l c}\hline
    ソフトウェア & バージョン\\\hline\hline
    Bazel & 0.29.1\\
    Tensorflow C++ API & 2.1.0\\
    CUDA & 10.1\\
    cuDNN & 7\\
    Eigen & 3.3.90\\
    Protobuf & 3.8\\
    g++ & 8.4.0\\
    iLCSoft & 02-02\\\hline
  \end{tabular}
  }
  \caption{崩壊点検出のソフトウェア動作環境}
  \label{SoftwareEnvironments}
\end{table}

このようにして実現された本研究の崩壊点検出を用いたフレーバータギングの性能と、LCFIPlusでの標準的なフレーバータギングの性能についての比較を\ref{Com:FlavorTaggingComparison}節で述べる。


%%%%%%%%%%%%%%%%%%%%%%%%%%%%%%%%%%%%%%%%%%%%%%%%%%%%%%%%%%%%%%%%%%%%%%%%%%%%%%%%%%%%%%%%%%%%%%%%%%%%%
\section{崩壊点検出単体での比較} \label{Com:ComparisonwithVF}

%%%%%%%%%%%%%%%%%%%%%%%%%%%%%%%%%%%%%%%%%%%%%%%%%%%%%%%%%%%%%%%%%%%%%%%%%%%%%%%%%%%%%%%%%%%%%%%%%%%%%
\section{より詳細な比較} \label{Com:FlavorTaggingComparison}

%%%%%%%%%%%%%%%%%%%%%%%%%%%%%%%%%%%%%%%%%%%%%%%%%%%%%%%%%%%%%%%%%%%%%%%%%%%%%%%%%%%%%%
\subsection{シングルトラックマージ} \label{Com:FlaTagCom:SingleTrackMerge}


更なる調査として、フレーバータギングまでを含めた性能の比較を行う必要がある。
ジェットの再構成は\ref{Intro:SoftERILC:JetReconstruction}項でも述べたように、崩壊点検出、ジェットクラスタリング、フレーバータギングの順で行われる。

LCFIPlus上で本研究で使用したネットワーク・崩壊点検出アルゴリズムを動作させ、次のジェットクラスタリングへ、その再構成情報を提供する為に一点アルゴリズムの変更を行った。
現在、LCFIPlusのジェットクラスタリングではLCFIPlusでのフィッティングによって得られる崩壊点の情報を使用している。
この崩壊点は二本以上の飛跡について定義されている。
しかし、本研究のネットワークでは再構成の過程によっては飛跡が一本しか含まれていないSVが生成されてしまう場合がある。
このような崩壊点についてジェットクラスタリングは上手く動作しない為、このような一本の飛跡についてはLCFIPlusのフィッティングを用いて、他の再構成されたSV内の飛跡の内、最も$\chi^2$の小さくなる飛跡対の組みを選び統合させている。
ただし、得られた$\chi^2$値が$400$を超えていた場合はこの飛跡を除外し、残りの飛跡と同様の取り扱いを行うこととした。

以上のような追加のアルゴリズムを含んだ崩壊点検出のプロセッサー (\ref{Intro:SoftERILC:Software}項) を作成し、現行の崩壊点検出と本研究の崩壊点検出を入れ替えたフレーバータギングの性能を計算した。
ただし、ジェットクラスタリングで使用される各種変数は現行の手法の崩壊点検出に最適化された値であり、本研究でのアルゴリズムに対しては不利な条件であると言える。

そのようにして得られたフレーバータギングによる$\rm b,\ c$タギングについてのROC曲線の結果を図\ref{}に示す。


%%%%%%%%%%%%%%%%%%%%%%%%%%%%%%%%%%%%%%%%%%%%%%%%%%%%%%%%%%%%%%%%%%%%%%%%%%%%%%%%%%%%%%
\subsection{LCFIPlus} \label{Com:FlaTagCom:LCFIPlus}

%%%%%%%%%%%%%%%%%%%%%%%%%%%%%%%%%%%%%%%%%%%%%%%%%%%%%%%%%%%%%%%%%%%%%%%%%%%%%%%%%%%%%%%%%%%%%%%%%%%%%
\section{フレーバータギングでの比較} \label{Com:FlaTagCom:FlavorTaggingComparison}




























