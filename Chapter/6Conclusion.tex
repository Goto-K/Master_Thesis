% !TEX root = ../MasterThesis_goto_v1.tex

\chapter{まとめと今後の展望} \label{chap:Conclusion}

本研究では, 深層学習を用いた崩壊点検出アルゴリズムの開発と実装を行った。
ジェットの再構成は, 加速器実験における新物理の手掛かりを探索する上で非常に重要な役割を果たしている。
ILCにおいては, ボトム・フレーバーとチャーム・フレーバーのジェットの高性能な分離が期待され, 現在はLCFIPlusのジェット再構成アルゴリズムが使用されている。
事象再構成アルゴリズムにはパターン認識技術が使用される場合も多く, 深層学習のような機械学習技術の発展による物理解析の精度向上が期待されている。
しかし, 崩壊点検出アルゴリズムについては深層学習を用いたアルゴリズムの研究は殆ど行われておらず, 本研究がその先駆けとなっている。

深層学習技術として一般的なフィードフォワードニューラルネットワークに加え, リカレントニューラルネットワークの一種であるLSTMを独自に拡張したネットワークやAttentionなど最新のテクニックを活用した。
崩壊点検出アルゴリズムの実現に当たって二種類のネットワーク「1. 飛跡対についてのネットワーク」, 「2. 任意の数の飛跡についてのネットワーク」をTensorflow/Kerasを用いて実装した。
ネットワークのための訓練データとして$\it e^+e^- \to b \bar{b}/c \bar{c}$事象のMCシミュレーションデータを使用し, 正解ラベルの作成にはMC情報やLCFIPlusのフィッターによる計算値を用いた。

これらのネットワークを組み合わせたアルゴリズムを作成し, 性能の評価を飛跡段階での効率を用いて行った。
性能の可否についてはLCFIPlusと比較する事によって評価し, まず崩壊点検出アルゴリズム単体での比較を行った後, ジェット再構成の最後段であるフレーバータギングでの比較を行った。
フレーバータギングの比較のため, 本ネットワーク・アルゴリズムのLCFIPlus, Marlinへの実装を行った。
本アルゴリズムはLCFIPlusのアルゴリズムと置換可能な状態であり, iLCSoft内での深層学習の活用について示すことができた。

崩壊点検出アルゴリズム単体での比較において, 本研究のアルゴリズムはsecondary vertexの再構成効率においてLCFIPlusのアルゴリズムを超えることができた (表\ref{PerformanceofAllEventsBB})。
一方で, 誤識別であるPrimary, Othersの値が悪化してしまっており単純な性能比較では性能の優劣を決めることが出来なかった。

フレーバータギングの比較において, 本研究で作成した崩壊点検出アルゴリズムはLCFIPlusの性能を超える事ができなかった。
これは, 深層学習で再構成された崩壊点に品質の悪い (誤差の大きな) 飛跡が混入してしまい雑音となってしまうからであると考えられる。
深層学習ではその様な誤差の情報を上手く取り入れる事が困難である為, 適切な損失関数やネットワークの作成が必要であるが, これらの研究は今後の課題としたい。\\

今後の展望として, 崩壊点検出アルゴリズムについては$2$点改善の余地が存在すると考えている。

一つは深層学習の理解からの改良である。
例えば, 現在, 「任意の飛跡についてのネットワーク」はエンコーダー・デコーダー部に対してリカレントニューラルネットワークの技術を使用しているが, リカレントニューラルネットワークは並列化しづらく, 学習・推論に時間がかかるという課題がある。
特にエンコーダー部にはより並列処理に互換性のある構造を取り入れる事でより高速な計算が可能になると考えている。
更に, 本研究では飛跡について空間方向の依存性を殆ど取り入れられていない為, 「畳み込みニューラルネットワーク」や「グラフニューラルネットワーク」などを活用する事によって改善が期待できる。

もう一つは物理的な性質からの改良である。
これは損失関数やネットワークの構造に対して, エネルギーなどの物理的性質を組み込む事による改善である。
本ネットワークはその様な物理的性質からの学習の方向付けを行なっておらず, 事象についての情報を十全に抽出できていなかったのではないかと考えられる。

本研究では, 残念ながらフレーバータギングの性能において従来の手法を上回ることが出来なかったが, これは深層学習としての利点を十分に活かせていない事もその要因として考えられる。
深層学習では中間層の情報の使用や転移学習といった様々な手段によって後段のアルゴリズムに情報を提供することが出来る。
今回の比較においては, 深層学習は単に推論にのみ使用し, この様な情報は全て捨ててしまっている為, 深層学習の真の価値を活かし切れたとは言いづらく, 後段のアルゴリズムを含めた全てのジェット再構成アルゴリズムを深層学習や機械学習に置き換える事でこれらの情報を活用することができると考えている。

















