% !TEX root = ../MasterThesis_goto_v1.tex

%%%%%%%%%%%%%%%%%%%%%%%%%%%%%%%%%%%%%%%%%%%%%%%%%%%%%%%%%%%%%%%%%%%%%%%%%%%%%%%%%%%%%%%%%%%%%%%%%%%%%
\chapter{崩壊点検出の為のネットワーク} \label{chap:Networks}

\ref{chap:Networks}章と\ref{chap:VertexFinderwithDL}章、\ref{chap:Comparison}章は本論文の本題である。
\ref{chap:Networks}章では本研究で使用するデータと深層学習のネットワークについて、\ref{chap:VertexFinderwithDL}章では\ref{chap:Networks}章で作成したネットワークを用いた崩壊点検出について、\ref{chap:Comparison}章ではその様にして作成した崩壊点検出と現行の崩壊点検出との比較について、それぞれ解説する。

本章では、まず\ref{Net:Data}節で本研究で取り扱うデータの性質について述べる。
次に、\ref{Net:forVertexFinderwithDL}節にて深層学習を使用した崩壊点検出をどのように実現するかについて概要を説明する。
個々のネットワークの詳細な構造や学習、ネットワーク単体での性能については\ref{Net:PairModel}節、\ref{Net:VertexLSTM}節で解説する。
これらネットワークについては\ref{chap:DeepLearning}章の内容を前提とし、訓練データ、ハイパーパラメーター・チューニングについてもここで述べる。

%%%%%%%%%%%%%%%%%%%%%%%%%%%%%%%%%%%%%%%%%%%%%%%%%%%%%%%%%%%%%%%%%%%%%%%%%%%%%%%%%%%%%%%%%%%%%%%%%%%%%
\section{データ} \label{Net:Data}

ここでは、本研究で使用したモンテカルロシミュレーションデータについて述べる。
特に\ref{Net:Data: MonteCarloSimulation}項では、データ生成についてのソフトウェアに関して、\ref{Net:Data:DataProperty}項では、データの性質について詳しく述べる。
ただし、各ネットワークの学習に使用する訓練データに関しては、後の\ref{Net:PM:TrainingandStrategyofPM}や\ref{Net:VLSTM:TrainingandStrategyofVLSTM}で紹介する。

%%%%%%%%%%%%%%%%%%%%%%%%%%%%%%%%%%%%%%%%%%%%%%%%%%%%%%%%%%%%%%%%%%%%%%%%
\subsection{モンテカルロシミュレーション} \label{Net:Data: MonteCarloSimulation}

どのようなシミュレーションデータセットを使っているか。\\

%%%%%%%%%%%%%%%%%%%%%%%%%%%%%%%%%%%%%%%%%%%%%%%%%%%%%%%%%%%%%%%%%%%%%%%%
\subsection{データの性質} \label{Net:Data:DataProperty}

データ全体について\\
エネルギー・終状態など\\
\\
各変数について\\
Helix parametersの元の分布やreshape後の分布について\\

%%%%%%%%%%%%%%%%%%%%%%%%%%%%%%%%%%%%%%%%%%%%%%%%%%%%%%%%%%%%%%%%%%%%%%%%%%%%%%%%%%%%%%%%%%%%%%%%%%%%%
\section{深層学習を用いた崩壊点検出の実現} \label{Net:forVertexFinderwithDL}

どのようなネットワークを使うか。\\
発想と役割、気をつけねばならないこと。\\

ソフトウェアやハードウェアに関してもここで述べる\\
tensorflowとか、TITANとか\\


%%%%%%%%%%%%%%%%%%%%%%%%%%%%%%%%%%%%%%%%%%%%%%%%%%%%%%%%%%%%%%%%%%%%%%%%%%%%%%%%%%%%%%%%%%%%%%%%%%%%%
\section{飛跡対についてのネットワーク} \label{Net:PairModel}

ここでは\ref{Net:forVertexFinderwithDL}節で紹介した二つのネットワークの内、飛跡対のためのネットワークについて述べる。
主にネットワークの構造に関しては\ref{Net:PM:StructureofPM}項で、学習に関しては\ref{Net:PM:TrainingandStrategyofPM}項で解説する。
また、そのようにして構築、訓練されたネットワーク単体についての性能と簡単な評価に関しては、\ref{Net:PM:PerformanceofPM}項で述べることとする。

%%%%%%%%%%%%%%%%%%%%%%%%%%%%%%%%%%%%%%%%%%%%%%%%%%%%%%%%%%%%%%%%%%%%%%%%
\subsection{ネットワークの構造} \label{Net:PM:StructureofPM}

%%%%%%%%%%%%%%%%%%%%%%%%%%%%%%%%%%%%%%%%%%%%%%%%%%%%%%%%%%%%%%%%%%%%%%%%
\subsection{ネットワークの学習と戦略} \label{Net:PM:TrainingandStrategyofPM}

%%%%%%%%%%%%%%%%%%%%%%%%%%%%%%%%%%%%%%%%%%%%%%%%%%%%%%%%%%%%%%%%%%%%%%%%
\subsection{ネットワークの性能} \label{Net:PM:PerformanceofPM}

%%%%%%%%%%%%%%%%%%%%%%%%%%%%%%%%%%%%%%%%%%%%%%%%%%%%%%%%%%%%%%%%%%%%%%%%%%%%%%%%%%%%%%%%%%%%%%%%%%%%%
\section{任意の数の飛跡についてのネットワーク} \label{Net:VertexLSTM}

ここでは\ref{Net:forVertexFinderwithDL}節で紹介した二つのネットワークの内、任意の数の飛跡のためのネットワークについて述べる。
基本的には前節の飛跡対についてのネットワークと同様の手順での解説を行うが、この任意の数の飛跡についてのネットワークは、既存のネットワーク構造にはない独自のネットワークで構築している。
これは本研究におけるデータの特殊性や問題解決のための最適なネットワークを考慮した結果である。
このようなネットワークの詳細な構造については\ref{Net:VLSTM:DetailedStructureofVLSTM}項で述べる。


%%%%%%%%%%%%%%%%%%%%%%%%%%%%%%%%%%%%%%%%%%%%%%%%%%%%%%%%%%%%%%%%%%%%%%%%
\subsection{ネットワークの構造} \label{Net:VLSTM:StructureofVLSTM}

%%%%%%%%%%%%%%%%%%%%%%%%%%%%%%%%%%%%%%%%%%%%%%%%%%%%%%%%%%%%%%%%%%%%%%%%
\subsection{ネットワークの詳細な構造} \label{Net:VLSTM:DetailedStructureofVLSTM}

%%%%%%%%%%%%%%%%%%%%%%%%%%%%%%%%%%%%%%%%%%%%%%%%%%%%%%%%%%%%%%%%%%%%%%%%
\subsection{ネットワークの学習と戦略} \label{Net:VLSTM:TrainingandStrategyofVLSTM}

%%%%%%%%%%%%%%%%%%%%%%%%%%%%%%%%%%%%%%%%%%%%%%%%%%%%%%%%%%%%%%%%%%%%%%%%
\subsection{ネットワークの性能} \label{Net:VLSTM:PerformanceofVLSTM}

